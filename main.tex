\documentclass[a4paper,12pt]{article}
\usepackage[T2A]{fontenc}
\usepackage[utf8]{inputenc}
\usepackage[main = russian,english]{babel}
\usepackage[14pt]{extsizes}
\usepackage{cmap}
\usepackage{indentfirst}
\usepackage{autonum}
\usepackage{amsfonts}
\usepackage{amsmath}
\usepackage{amssymb}
\usepackage{amsthm}
\usepackage{upgreek}
\usepackage{graphicx}
\usepackage{listings}
\usepackage{multirow}
\usepackage{setspace,amsmath}
\usepackage[table,xcdraw]{xcolor}
\usepackage[left=15mm, top=20mm, right=15mm, bottom=20mm, nohead, footskip=15mm]{geometry}

\begin{document}
\setcounter{page}{-1}

\begin{center}
	\small{Министерство науки и высшего образования Российской Федерации}\\
	\small{Федеральное государственное бюджетное образовательное учреждение}\\
	\small{Высшего образования}\\
	\small{\textbf{«Северо-Осетинский государственный университет\\
			имени Коста Левановича Хетагурова»}}\\
		
	\hfill \break
	\hfill \break
	\hfill \break
	\hfill \break
	\hfill \break
	\hfill \break
	\hfill \break
	\hfill \break
	\hfill \break
	
	\normalsize{Курсовая работа}\\
	\large{\textbf{«Транспортная задача. Методы решенияю.»}}\\
	
	\hfill \break
	\hfill \break
	\hfill \break
	\hfill \break
	\hfill \break
	\hfill\break
\end{center}

\begin{flushright}
	\textbf{Выполнил:}\\
	Студент 3 курса направления:\\
	«Прикладная математика и информатика»\\
	\textit{Гамосов Станислав Станиславович}\\
\end{flushright}

\hfill

\begin{flushright}
	\textbf{Научный руководитель:}\\
	кандидат физико-математических наук\\
	\textit{Тотиева Жанна Дмитриевна}\\
\end{flushright}

\hfill

\begin{flushright}
	\textbf{«Работа допустима к защите»}\\
	Заведующий кафедрой\\
	доктор физико-математических наук\\
	\textit{Кусраев. А.Г. \underline{\hspace{3cm}}}\\
\end{flushright}

\normalsize{ \hspace{28pt}} \hfill \break
\begin{center} Владикавказ 2021 \end{center}

\thispagestyle{empty}
\tableofcontents{}
\thispagestyle{empty}
\clearpage

\section{Введение}
	\textbf{Транспортная задача} – это спектр задач с единой математической моделью, классическая формулировка, которой звучит: \textit{«Задача о наиболее экономном плане перевозок однородного продукта или взаимозаменяемых продуктов из пунктов производства в пункты потребления»}. Такая форма встречается чаще всего в линейном программирование, а если точнее в его практических приложениях. 
	
	\textbf{Линейное программирование} является одним из разделов математического программирования – области математики, разрабатывающей теорию и численные методы решения многомерных экстремальных задач с ограничениями.
	
	Проблема была впервые формализована французским математиком \textit{Гаспаром Монжем} в 1781 году. Прогресс в решении проблемы был достигнут во время Великой Отечественной войны советским математиком и экономистом \textit{Леонидом Канторовичем}. Поэтому иногда эта проблема называется \textbf{транспортной задачей Монжа — Канторовича}.
	
	Если вернуться к самой задачи огромное количество возможных вариантов перевозок затрудняет получение достаточно экономного плана эмпирическим или экспертным путем. Применение математических методов и вычислительных в планировании перевозок дает большой экономический эффект. Транспортные задачи могут быть решены \textbf{симплексным методом} однако матрица системы ограничений транспортной задачи настолько своеобразна, что для ее решения разработаны специальные методы. Они, как и \textbf{симплексный метод}, позволяют найти начальное опорное решение, а затем, улучшая его получить оптимальный результат. Транспортная задача может
	также решаться с ограничениями и без ограничений.
	
	В зависимости от способа представления условий транспортной задачи она может быть представлена в \textbf{графовой} или \textbf{матричной} форме.
	
	\clearpage
	
\section{Постановка задачи}
	
	Задача эта возникает, когда речь идет о рациональной перевозке некоторого однородного продукта от производителей к потребителям. В этом случае для каждого потребителя безразлично, откуда, из каких пунктов производства будет поступать этот продукт, лишь бы он поступал в нужном объеме. Однако от того, насколько рациональным будет прикрепление пунктов потребления к пунктам производства, существенно зависит объем транспортной работы. В связи с этим естественно возникает вопрос о наиболее эффективном прикреплении, правильном направлении перевозок груза, при котором потребности удовлетворяются, а затраты на транспортировку минимальны. Более точно задача формулируется так.
	
	Пусть имеются пункты производства $(A_1, A_2, ...,  A_n)$ с объемами производства в единицу времени, равными соответственно $(a_1, a_2, ...,  a_n)$, и пункты потребления $(B_1, B_2, ...,  B_m)$ с объемами потребления, равными $(b_1, b_2, ..., b_m)$ соответственно. Будем предполагать, что производство и потребление сбалансированы — сумма объемов производства равна сумме объемов потребления. Такой вид транспортной задачи называется \textbf{закрытым}.
	
	\[
		\sum\limits_{i=1}^n a_i = \sum\limits_{j=1}^m b_j \eqno(1)
	\]
	
	В дальнейшем будем рассматривать только такой тип задачи. Однако любую \textbf{открытую} транспортную задачу ($\sum\limits_{i=1}^n a_i \neq \sum\limits_{j=1}^m b_j$) легко закрыть. Нужно ввести дополнительный пункт производства (пункт потребления) с недостающим объемом производства (объемом потребления) и с нулевыми стоимостями перевозок.

	Предполагается, что известны величины $c_{ij}$ — затраты по перевозке единицы продукта из $i$-го пункта производства в $j$-й пункт потребления. Они могут быть выражены в стоимостной (денежной) форме или в натуральной (километрах). Требуется найти такой план перевозок, при котором были бы удовлетворены потребности в пунктах $(B_1, B_2, ...,  B_m)$ и при этом суммарные затраты на перевозку были бы минимальны. Обозначая через $x_{ij}$ количество продукта, перевозимое из $i$-го пункта производства в $j$-го пункт потребления, приходим к следующей математической формулировке задачи:
	
	\textbf{Найти минимум целевой функции}:
	
	\[
		\sum\limits_{i=1}^n \sum\limits_{j=1}^m c_{ij}x_{ij} \longrightarrow \min \eqno(2)
	\]

	\newpage
	Так же для корректности задачи необходимо соблюдать три условия:
	
	\[
		\textbf{1.} \sum\limits_{i=1}^n x_{ij} = b_j; (j = 1, 2, ..., m) 
	\]
	
	\[
		\textbf{2.} \sum\limits_{j=1}^m x_{ij} = a_i; (i = 1, 2, ..., n)
	\]
	
	\[
		\textbf{3.} x_{ij} \geqslant 0; (i = 1, 2, ..., n; j = 1, 2, ..., m)
	\]

	Получается суммарные затраты на транспортировку в каждый пункт потребления завозится требуемое количество продукта, а так же из каждого пункта производства полностью вывозится произведенный продукт.
	
	Всякий набор величин $x_{ij} (i = 1, 2, ..., n; j = 1, 2, ..., m)$, удовлетворяющих условиям $(1-3)$, мы будем называть допустимым планом перевозок. План, для которого суммарные затраты $(2)$ достигают минимума, называется оптимальным.
	
	\begin{equation}
		X = 
		\begin{pmatrix}
			x_{1,1} & x_{1,2} & \cdots & x_{1,n} \\
			x_{2,1} & x_{2,2} & \cdots & x_{2,n} \\
			\vdots  & \vdots  & \ddots & \vdots  \\
			x_{m,1} & x_{m,2} & \cdots & x_{m,n} 
		\end{pmatrix}
	\end{equation}

	\begin{flushright}
		\textit{Таблица 2.1}
	\end{flushright}
	\begin{center}
		\begin{tabular}{|c|c|c|c|c|c|c|c|}
			\hline
			\multirow{2}{*}{Поставщики} & \multicolumn{6}{c|}{Потребители} & \multirow{2}{*}{\begin{tabular}[c]{@{}c@{}}Запасы \\ поставщика\end{tabular}} \\ \cline{2-7}
			& 1 & 2 & $\cdots$ & $j$ & $\cdots$ & $m$ &  \\ \hline
			1 & $\begin{matrix} c_{11} & \\ & x_{11} \end{matrix}$ & $\begin{matrix} c_{12} & \\ & x_{12} \end{matrix}$ & $\cdots$  & $\begin{matrix} c_{1j} & \\ & x_{1j} \end{matrix}$  & $\cdots$ & $\begin{matrix} c_{1m} & \\ & x_{1m} \end{matrix}$  & $a_1$  \\ \hline
			2 & $\begin{matrix} c_{21} & \\ & x_{21} \end{matrix}$ & $\begin{matrix} c_{22} & \\ & x_{22} \end{matrix}$  & $\cdots$ & $\begin{matrix} c_{2j} & \\ & x_{2j} \end{matrix}$  & $\cdots$ & $\begin{matrix} c_{2m} & \\ & x_{2m} \end{matrix}$  & $a_2$  \\ \hline
			$\vdots$ & $\vdots$ & $\vdots$ & $\cdots$ & $\vdots$ & $\cdots$ & $\vdots$ & $\vdots$ \\ \hline
			$i$ & $\begin{matrix} c_{i1} & \\ & x_{i1} \end{matrix}$ & $\begin{matrix} c_{i2} & \\ & x_{i2} \end{matrix}$ & $\cdots$  & $\begin{matrix} c_{ij} & \\ & x_{ij} \end{matrix}$ & $\cdots$ & $\begin{matrix} c_{im} & \\ & x_{im} \end{matrix}$ & $a_i$ \\ \hline
			$\vdots$ & $\vdots$ & $\vdots$ & $\cdots$ & $\vdots$ & $\cdots$ & $\vdots$ & $\vdots$ \\ \hline
			$n$ & $\begin{matrix} c_{n1} & \\ & x_{n1} \end{matrix}$ & $\begin{matrix} c_{n2} & \\ & x_{n2} \end{matrix}$ & $\cdots$  & $\begin{matrix} c_{nj} & \\ & x_{nj} \end{matrix}$ & $\cdots$ & $\begin{matrix} c_{nm} & \\ & x_{nm} \end{matrix}$ & $a_n$  \\ \hline
			\begin{tabular}[c]{@{}c@{}}Спрос \\ потребителя\end{tabular} & $b_1$ & $b_2$ & $\cdots$ & $b_j$ & $\cdots$ & $b_m$ & $\begin{matrix} & \sum\limits_{i=1}^n a_i \\ \sum\limits_{j=1}^m b_j & \end{matrix}$ \\ \hline
		\end{tabular}
	\end{center}

	Рассмотрим теорему о разрешимости транспортной задачи:
	
	\newtheorem{theorem}{Теорема}
	\begin{theorem}
		Транспортная задача имеет решение, если суммарный запас груза в пунктах отправления равен суммарному спросу в пунктах назначения, т.е. если выполняется равенство (1).
	\end{theorem}
	
	\begin{proof}
		В случае превышения запаса над потребностью
		$
			\sum\limits_{i=1}^n a_i > \sum\limits_{j=1}^m b_j
		$
		как уже было обговорено выше, вводится фиктивный $(m + 1)$-ый пункт назначения с потребностью
		$
			b_{m + 1} = \sum\limits_{i = 1}^n a_i - \sum\limits_{j = 1}^m b_j
		$
		
		Соответствующие тарифы считаются равными нулю: 
		
		\[
			c_{i m+1} = 0 (i=1, \cdots ,m)
		\]
		
		После этих преобразований получим закрытую модель транспортной задачи.
		
		Аналогично, при
		$
			\sum\limits_{i=1}^n a_i < \sum\limits_{j=1}^m b_j
		$
		вводится фиктивный $(n+1)$ пункт отправления с грузом,
		$
			a_{n + 1} = \sum\limits_{j = 1}^m b_j - \sum\limits_{i = 1}^n a_i
		$
		а тарифы полагаются равными нулю:
		
		\[ 
			c_{n+1 j} = 0 (j=1, \cdots ,n)
		\]
		 
		После этих преобразований получим закрытую модель транспортной задачи.
	\end{proof}

	Полученные условия транспортной задачи удобно представить в виде матрицы, которая имеет название \textbf{матрица перевозок}. В первой строке указаны величины потребностей, в первом столбце - значения запасов. В клетках внутренней матрицы ($m \times n$ штук) записывают стоимости перевозок и сами перевозки. Нумеровать будем только строки и столбцы внутренней матрицы.
	
	\textbf{Пример:}
	
	Составить математическую модель транспортной задачи перевоза груза из 3 складов в 5 магазинов. Матрица перевозок будет выглядеть так:
	\begin{flushright}
		\textit{Таблица 2.2}
	\end{flushright}
	\begin{center}
		\begin{tabular}{c|cccccc}
			\cline{2-7}
			& \multicolumn{1}{c|}{$b_j$} & \multicolumn{1}{c|}{10} & \multicolumn{1}{c|}{40} & \multicolumn{1}{c|}{20} & \multicolumn{1}{c|}{60} & \multicolumn{1}{c|}{20} \\ \hline
			\multicolumn{1}{|c|}{$a_i$} &  &  &  &  &  &  \\ \cline{1-1} \cline{3-7} 
			\multicolumn{1}{|c|}{30} & \multicolumn{1}{c|}{} & \multicolumn{1}{c|}{2} & \multicolumn{1}{c|}{7} & \multicolumn{1}{c|}{3} & \multicolumn{1}{c|}{6} & \multicolumn{1}{c|}{2} \\ \cline{1-1} \cline{3-7} 
			\multicolumn{1}{|c|}{70} & \multicolumn{1}{c|}{} & \multicolumn{1}{c|}{9} & \multicolumn{1}{c|}{4} & \multicolumn{1}{c|}{5} & \multicolumn{1}{c|}{7} & \multicolumn{1}{c|}{3} \\ \cline{1-1} \cline{3-7} 
			\multicolumn{1}{|c|}{50} & \multicolumn{1}{c|}{} & \multicolumn{1}{c|}{5} & \multicolumn{1}{c|}{7} & \multicolumn{1}{c|}{6} & \multicolumn{1}{c|}{2} & \multicolumn{1}{c|}{4} \\ \cline{1-1} \cline{3-7} 
		\end{tabular}
	\end{center}

	\[
		 \sum\limits_{i=1}^3 a_{i} = 30 + 70 + 50 = 150 \ \ \ \
		 \sum\limits_{j=1}^5 b_{j} = 10 + 40 + 20 + 60 + 20 = 150 
	\]
	
	В качестве примера открытой модели давайте заменим потребность $b_4$, которая равняется 60 на 40. В таком случаи нужно было бы ввести еще одного потребителя с потребностью $b_6 = 20$ и с нулевыми стоимостями $c_{16} = c_{26} = c_{36} = 0$. Матрица перевозок тогда станет следующей:
	
	\begin{flushright}
		\textit{Таблица 2.3}
	\end{flushright}
	\begin{center}
		\begin{tabular}{c|ccccccc}
			\cline{2-8}
			& \multicolumn{1}{c|}{$b_j$} & \multicolumn{1}{c|}{10} & \multicolumn{1}{c|}{40} & \multicolumn{1}{c|}{20} & \multicolumn{1}{c|}{40} & \multicolumn{1}{c|}{20} & \multicolumn{1}{l|}{20} \\ \hline
			\multicolumn{1}{|c|}{$a_i$} &  &  &  &  &  &  & \multicolumn{1}{l}{} \\ \cline{1-1} \cline{3-8} 
			\multicolumn{1}{|c|}{30} & \multicolumn{1}{c|}{} & \multicolumn{1}{c|}{2} & \multicolumn{1}{c|}{7} & \multicolumn{1}{c|}{3} & \multicolumn{1}{c|}{6} & \multicolumn{1}{c|}{2} & \multicolumn{1}{c|}{0} \\ \cline{1-1} \cline{3-8} 
			\multicolumn{1}{|c|}{70} & \multicolumn{1}{c|}{} & \multicolumn{1}{c|}{9} & \multicolumn{1}{c|}{4} & \multicolumn{1}{c|}{5} & \multicolumn{1}{c|}{7} & \multicolumn{1}{c|}{3} & \multicolumn{1}{c|}{0} \\ \cline{1-1} \cline{3-8} 
			\multicolumn{1}{|c|}{50} & \multicolumn{1}{c|}{} & \multicolumn{1}{c|}{5} & \multicolumn{1}{c|}{7} & \multicolumn{1}{c|}{6} & \multicolumn{1}{c|}{2} & \multicolumn{1}{c|}{4} & \multicolumn{1}{c|}{0} \\ \cline{1-1} \cline{3-8} 
		\end{tabular}
	\end{center}

	Запишем пример закрытой модели в полном виде:
	
	\begin{equation}
		\begin{cases}
			Z = \sum\limits_{i=1}^n \sum\limits_{j=1}^m c_{ij}x_{ij} \longrightarrow \min \\
			x_{11} + x_{12} + x_{13} + x_{14} + x_{15} = 30\\
			x_{21} + x_{22} + x_{23} + x_{24} + x_{25} = 70\\
			x_{31} + x_{32} + x_{33} + x_{34} + x_{35} = 50\\
			x_{11} + x_{21} + x_{31} = 10\\
			x_{12} + x_{22} + x_{32} = 40\\
			x_{13} + x_{23} + x_{33} = 20\\
			x_{14} + x_{24} + x_{34} = 60\\
			x_{15} + x_{25} + x_{35} = 20\\
			x_{11} + x_{21} + x_{31} = 10\\ 
			\forall (i,j): x_{ij} \geqslant 0
		\end{cases}
	\end{equation}
	\clearpage

\section{Методы построения опорного решения}
	Прежде чем исследовать методы нахождения опорного решения транспортной задачи, рассмотрим теорему о ранге матрицы из коэффициентов при неизвестных системы ее ограничений.
	
	\begin{theorem}
		Ранг матрицы из коэффициентов при неизвестных системы ограничений транспортной задачи равен $(m + n - 1)$, где $m$ и $n$ - количество поставщиков и потребителей соответственно.
	\end{theorem}
	
	Каждую неизвестную запишем в свой столбец и получим такую систему:
	\begin{equation}
		\begin{Bmatrix}
			x_{11} + x_{12} + \cdots + x_{1m} & & & = a_1\\
			& x_{21} + x_{22} + \cdots + x_{2m} & & = a_2\\
			\cdots & \cdots & \cdots & & \\
			& & x_{n1} + x_{n2} + \cdots + x_{nm} & = a_n\\
			\hline
			x_{11} & + x_{21} & + x_{n1} & = b_1 \\
			x_{12} & + x_{22} & + x_{n2} & = b_2 \\
			\cdots & \cdots & \cdots & & \\
			x_{1n} & + x_{2n} & + x_{nm} & = b_m \\
		\end{Bmatrix}
	\end{equation}

	Матрица $A$ их коэффициентов при неизвестных системы ограничений имеет вид:
	
	\begin{equation}
		A = 
		\begin{bmatrix}
			11 \ldots 1 & 00 \ldots 0 & 00 \ldots 0 \\
			00 \ldots 0 & 11 \ldots 1 & 00 \ldots 0 \\
			\ldots & \ldots & \ldots \\
			00 \ldots 0 & 00 \ldots 0 & 11 \ldots 1 \\
			10 \ldots 0 & 10 \ldots 0 & 10 \ldots 0 \\
			01 \ldots 0 & 01 \ldots 0 & 01 \ldots 0 \\
			\ldots & \ldots & \ldots \\
			00 \ldots 1 & 00 \ldots 1 & 00 \ldots 1 \\
		\end{bmatrix}
	\end{equation}
	
	Нетрудно видеть, что любую строку матрицы можно выразить в виде линейной комбинации остальных ее строк. Сложив коэффициенты первых $n$ строк матрицы $A$, a потом коэффициенты последующих $m - 1$ строк и вычтя из первой суммы вторую, получим элементы последней строки матрицы. Отсюда заключаем, что ранг матрицы $А$ не будет изменяться при вычеркивании одной какой-нибудь ее строки.
	
	Вычеркнем последнюю строку матрицы $A$ и вычислим минор $M_{n + m - 1}$ , образованный из $(n + m - 1)$ столбцов коэффициентов при следующих неизвестных: $x_{1m}, x_{2m}, \ldots, x_{mn}, x_{11}, x_{12}, \ldots, x_{1, m - 1}$
	
	\begin{equation}
		M_{n + m - 1} = 
		\begin{bmatrix}
			10 \ldots 0 & 11 \ldots 1\\
			01 \ldots 0 & 00 \ldots 0\\
			\ldots & \ldots\\
			00 \ldots 1 & 00 \ldots 0\\
			10 \ldots 0 & 10 \ldots 0\\
			01 \ldots 0 & 01 \ldots 0\\
			\ldots & \ldots\\
			00 \ldots 0 & 00 \ldots 1\\
		\end{bmatrix}
		= 1
	\end{equation}
	
	Так как минор не равен нулю, то количество линейно независимых строк в матрице $A$ будет $n + m - 1$, т.е. ранг матрицы $A$ равен $n + m - 1$.
	
	Из рассмотренной теоремы следует, что опорное решение транспортной задачи должно содержать $n + m - 1$ базисных неизвестных и $nm - (n + m - 1)$ небазисных, равных нулю неизвестных.
	
	Циклом, или замкнутым контуром, называется последовательность клеток $(i, j)$ транспортной задачи, в которой каждые две рядом стоящие клетки находятся в одной строке или одном столбце, при этом первая и последняя	клетки совпадают.
	
	Например, $\nu = [(1,2);(1,4);(3,4);(3,2);(1,2)]$ есть цикл.
	
	\begin{center}
		\begin{tabular}{|c|c|c|c|c|}
			\hline
			$\begin{matrix} & j \\ i & \\ \end{matrix}$ & 1 & 2 & 3 & 4 \\ \hline
			1 &  & $\odot$ & --- & $\odot$ \\ \hline
			2 &  & $|$ &     & $|$ \\ \hline
			3 &  & $\odot$ & --- & $\odot$ \\ \hline
		\end{tabular}
	\end{center}
	
	\clearpage

	\begin{thebibliography}{3}
		\bibitem{1}
		Юдин Д.Б., Гольштейн Е.Г. Задачи и методы линейного программирования;
		Москва: Советское радио, 1969. - 736с.
		\bibitem{2}
		Канторович Л.В., Горстко А.Б. Математическое оптимальное программирование  в экономике;
		Москва: Знание, 1968. - 96с.
		\bibitem{3}
		Палий И.А., Линейное программирование. Учебное пособие;
		Москва: Эксмо, 2008. - 256с.
		\bibitem{4} 
		Костевич Л.C. Математическое программирование. Информационные технологии оптимальных решений;
		Минск: Новое знание, 2003. - 424c.
		\bibitem{5}
		Акулич И.Л. Математическое программирование в примерах и задачах; Санкт-Петербург: Лань, 2011 - 352с.
		\bibitem{6}
		Данциг Д. Линейное программирование, его обобщения и применения; Москва: Прогресс, 1966 - 602с.
		\bibitem{7}
	\end{thebibliography}

\end{document} 