\documentclass[a4paper]{report}
\usepackage[T2A]{fontenc}
\usepackage[utf8]{inputenc}
\usepackage[main = russian,english]{babel}
\usepackage[14pt]{extsizes}
\usepackage{cmap}
\usepackage{indentfirst}
\usepackage{autonum}
\usepackage{amsfonts}
\usepackage{amsmath}
\usepackage{amssymb}
\usepackage{amsthm}
\usepackage{upgreek}
\usepackage{graphicx}
\usepackage{listings}
\usepackage{setspace,amsmath}
\usepackage[left=15mm, top=20mm, right=15mm, bottom=20mm, nohead, footskip=15mm]{geometry}

\begin{document}
\setcounter{page}{0}

\begin{center}
	\small{Министерство науки и высшего образования Российской Федерации}\\
	\small{Федеральное государственное бюджетное образовательное учреждение}\\
	\small{Высшего образования}\\
	\small{\textbf{«Северо-Осетинский государственный университет\\
			имени Коста Левановича Хетагурова»}}\\
		
	\hfill \break
	\hfill \break
	\hfill \break
	\hfill \break
	\hfill \break
	\hfill \break
	\hfill \break
	\hfill \break
	\hfill \break
	
	\normalsize{Курсовая работа}\\
	\large{\textbf{«Транспортная задача. Методы решенияю.»}}\\
	
	\hfill \break
	\hfill \break
	\hfill \break
	\hfill \break
	\hfill \break
	\hfill\break
\end{center}

\begin{flushright}
	\textbf{Выполнил:}\\
	Студент 3 курса направления:\\
	«Прикладная математика и информатика»\\
	\textit{Гамосов Станислав Станиславович}\\
\end{flushright}

\hfill

\begin{flushright}
	\textbf{Научный руководитель:}\\
	кандидат физико-математических наук\\
	\textit{Тотиева Жанна Дмитриевна}\\
\end{flushright}

\hfill

\begin{flushright}
	\textbf{«Работа допустима к защите»}\\
	Заведующий кафедрой\\
	доктор физико-математических наук\\
	\textit{Кусраев. А.Г. \underline{\hspace{3cm}}}\\
\end{flushright}

\normalsize{ \hspace{28pt}} \hfill \break
\begin{center} Владикавказ 2021 \end{center}
\thispagestyle{empty}


\tableofcontents{}
\clearpage

\chapter{Введение}
	\textbf{Транспортная задача} – это спектр задач с единой математической моделью, классическая формулировка, которой звучит: \textit{«Задача о наиболее экономном плане перевозок однородного продукта или взаимозаменяемых продуктов из пунктов производства в пункты потребления»}. Такая форма встречается чаще всего в линейном программирование, а если точнее в его практических приложениях. 
	
	\textbf{Линейное программирование} является одним из разделов математического программирования – области математики, разрабатывающей теорию и численные методы решения многомерных экстремальных задач с ограничениями.
	
	Если вернуться к самой задачи огромное количество возможных вариантов перевозок затрудняет получение достаточно экономного плана эмпирическим или экспертным путем. Применение математических методов и вычислительных в планировании перевозок дает большой экономический эффект. Транспортные задачи могут быть решены \textbf{симплексным методом} однако матрица системы ограничений транспортной задачи настолько своеобразна, что для ее решения разработаны специальные методы. Они, как и \textbf{симплексный метод}, позволяют найти начальное опорное решение, а затем, улучшая его получить оптимальный результат. Транспортная задача может
	также решаться с ограничениями и без ограничений.
	
	В зависимости от способа представления условий транспортной задачи она может быть представлена в \textbf{графовой} или \textbf{матричной} форме.
	
\chapter{Постановка задачи}
	
	Задача эта возникает, когда речь идет о рациональной перевозке некоторого однородного продукта от производителей к потребителям. В этом случае для каждого потребителя безразлично, откуда, из каких пунктов производства будет поступать этот продукт, лишь бы он поступал в нужном объеме. Однако от того, насколько рациональным будет прикрепление пунктов потребления к пунктам производства, существенно зависит объем транспортной работы. В связи с этим естественно возникает вопрос о наиболее эффективном прикреплении, правильном направлении перевозок груза, при котором потребности удовлетворяются, а затраты на транспортировку минимальны. Более точно задача формулируется так.
	
	Пусть имеются пункты производства $(A_1, A_2, ...,  A_n)$ с объемами производства в единицу времени, равными соответственно $(a_1, a_2, ...,  a_n)$, и пункты потребления $(B_1, B_2, ...,  B_m)$ с объемами потребления, равными $(b_1, b_2, ..., b_m)$ соответственно. Будем предполагать, что производство и потребление сбалансированы — сумма объемов производства равна сумме объемов потребления
	
	\begin{center}
		$\sum\limits_{i=1}^n a_i = \sum\limits_{j=1}^m b_j$
	\end{center}

	Предполагается, что известны величины $C_{ij}$ — затраты по перевозке единицы продукта из $i$-го пункта производства в $j$-й пункт потребления. Они могут быть выражены в стоимостной (денежной) форме или в натуральной (километрах). Требуется найти такой план перевозок, при котором были бы удовлетворены потребности в пунктах $(B_1, B_2, ...,  B_m)$ и при этом суммарные затраты на перевозку были бы минимальны. Обозначая через $x_{ij}$ количество продукта, перевозимое из $i$-го пункта производства в $j$-го пункт потребления, приходим к следующей математической формулировке задачи:
	
	Найти минимум
	
	\[
		\sum\limits_{i=1}^n \sum\limits_{j=1}^m C_{ij}x_{ij}
		\label{eq:ref}
	\]

	Так же для корректности задачи необходимо соблюдать три условия:
	
	\begin{enumerate} 
		\item $\sum\limits_{i=1}^n x_{ij} = b_j, (j = 1, 2, ..., m)$
		\item $\sum\limits_{j=1}^m x_{ij} = a_i, (i = 1, 2, ..., n)$
		\item $x_{ij} \geqslant 0, (i = 1, 2, ..., n; j = 1, 2, ..., m)$ 
	\end{enumerate}

	Получается суммарные затраты на транспортировку в каждый пункт потребления завозится требуемое количество продукта, а так же из каждого пункта производства полностью вывозится произведенный продукт.
	
	Всякий набор величин $x_{ij} (i = 1, 2, ..., n; j = 1, 2, ..., m)$, удовлетворяющих условиям $(1-3)$, мы будем называть допустимым планом перевозок. План, для которого суммарные затраты (\ref{eq:ref}) достигают минимума, называется оптимальным.
	
\begin{thebibliography}{3}
	\bibitem{1}
	Юдин Д.Б., Гольштейн Е.Г. Задачи и методы линейного
	программирования. – М.: Советское радио, 1969. – 736с.
	\bibitem{2}
\end{thebibliography}

\end{document}