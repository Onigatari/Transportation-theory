\documentclass[a4paper,12pt]{article}
\usepackage[T2A]{fontenc}
\usepackage[utf8]{inputenc}
\usepackage[russian,english]{babel}
\usepackage[14pt]{extsizes}
\usepackage{cmap}
\usepackage{indentfirst}
\usepackage{autonum}
\usepackage{amsfonts}
\usepackage{amsmath}
\usepackage{amssymb}
\usepackage{amsthm}
\usepackage{upgreek}
\usepackage{graphicx}
\usepackage{listings}
\usepackage{multirow}
\usepackage{setspace,amsmath}
\usepackage[table,xcdraw]{xcolor}
\usepackage[unicode, pdftex]{hyperref}
\usepackage[left=15mm, top=20mm, right=15mm, bottom=20mm, nohead, footskip=15mm]{geometry}


\begin{document}
\selectlanguage{russian}
\setcounter{page}{-1}

\begin{center}
	\small{Министерство науки и высшего образования Российской Федерации}\\
	\small{Федеральное государственное бюджетное образовательное учреждение}\\
	\small{Высшего образования}\\
	\small{\textbf{«Северо-Осетинский государственный университет\\
			имени Коста Левановича Хетагурова»}}\\
		
	\hfill \break
	\hfill \break
	\hfill \break
	\hfill \break
	\hfill \break
	\hfill \break
	\hfill \break
	\hfill \break
	\hfill \break
	
	\normalsize{Курсовая работа}\\
	\large{\textbf{«Транспортная задача. Методы решенияю.»}}\\
	
	\hfill \break
	\hfill \break
	\hfill \break
	\hfill \break
	\hfill \break
	\hfill\break
\end{center}

\begin{flushright}
	\textbf{Выполнил:}\\
	Студент 3 курса направления:\\
	«Прикладная математика и информатика»\\
	\textit{Гамосов Станислав Станиславович}\\
\end{flushright}

\hfill

\begin{flushright}
	\textbf{Научный руководитель:}\\
	кандидат физико-математических наук\\
	\textit{Тотиева Жанна Дмитриевна}\\
\end{flushright}

\hfill

\begin{flushright}
	\textbf{«Работа допустима к защите»}\\
	Заведующий кафедрой\\
	доктор физико-математических наук\\
	\textit{Кусраев. А.Г. \underline{\hspace{3cm}}}\\
\end{flushright}

\normalsize{ \hspace{28pt}} \hfill \break
\begin{center} Владикавказ 2021 \end{center}

\thispagestyle{empty}
\tableofcontents
\thispagestyle{empty}
\clearpage
\newtheorem{theorem}{Теорема}

\section {Введение}
	\textbf{Транспортная задача} – это спектр задач с единой математической моделью, классическая формулировка, которой звучит: \textit{«Задача о наиболее экономном плане перевозок однородного продукта или взаимозаменяемых продуктов из пунктов производства в пункты потребления»}. Такая форма встречается чаще всего в линейном программирование, а если точнее в его практических приложениях. 
	
	\textbf{Линейное программирование} является одним из разделов математического программирования – области математики, разрабатывающей теорию и численные методы решения многомерных экстремальных задач с ограничениями.
	
	Проблема была впервые формализована французским математиком \textit{Гаспаром Монжем} в 1781 году. Прогресс в решении проблемы был достигнут во время Великой Отечественной войны советским математиком и экономистом \textit{Леонидом Канторовичем}. Поэтому иногда эта проблема называется \textbf{транспортной задачей Монжа — Канторовича}.
	
	Если вернуться к самой задачи огромное количество возможных вариантов перевозок затрудняет получение достаточно экономного плана эмпирическим или экспертным путем. Применение математических методов и вычислительных в планировании перевозок дает большой экономический эффект. Транспортные задачи могут быть решены \textbf{симплексным методом} однако матрица системы ограничений транспортной задачи настолько своеобразна, что для ее решения разработаны специальные методы. Они, как и \textbf{симплексный метод}, позволяют найти начальное опорное решение, а затем, улучшая его получить оптимальный результат. Транспортная задача может
	также решаться с ограничениями и без ограничений.
	
	В зависимости от способа представления условий транспортной задачи она может быть представлена в \textbf{графовой} или \textbf{матричной} форме.
	
	\clearpage
	
\section{Постановка задачи}
	
	Задача эта возникает, когда речь идет о рациональной перевозке некоторого однородного продукта от производителей к потребителям. В этом случае для каждого потребителя безразлично, откуда, из каких пунктов производства будет поступать этот продукт, лишь бы он поступал в нужном объеме. Однако от того, насколько рациональным будет прикрепление пунктов потребления к пунктам производства, существенно зависит объем транспортной работы. В связи с этим естественно возникает вопрос о наиболее эффективном прикреплении, правильном направлении перевозок груза, при котором потребности удовлетворяются, а затраты на транспортировку минимальны. Более точно задача формулируется так.
	
	Пусть имеются пункты производства $(A_1, A_2, ...,  A_n)$ с объемами производства в единицу времени, равными соответственно $(a_1, a_2, ...,  a_n)$, и пункты потребления $(B_1, B_2, ...,  B_m)$ с объемами потребления, равными $(b_1, b_2, ..., b_m)$ соответственно. Будем предполагать, что производство и потребление сбалансированы — сумма объемов производства равна сумме объемов потребления. Такой вид транспортной задачи называется \textbf{закрытым}.
	
	\[
		\sum\limits_{i=1}^n a_i = \sum\limits_{j=1}^m b_j \eqno(1)
	\]
	
	В дальнейшем будем рассматривать только такой тип задачи. Однако любую \textbf{открытую} транспортную задачу ($\sum\limits_{i=1}^n a_i \neq \sum\limits_{j=1}^m b_j$) легко закрыть. Нужно ввести дополнительный пункт производства (пункт потребления) с недостающим объемом производства (объемом потребления) и с нулевыми стоимостями перевозок.

	Предполагается, что известны величины $c_{ij}$ — затраты по перевозке единицы продукта из $i$-го пункта производства в $j$-й пункт потребления. Они могут быть выражены в стоимостной (денежной) форме или в натуральной (километрах). Требуется найти такой план перевозок, при котором были бы удовлетворены потребности в пунктах $(B_1, B_2, ...,  B_m)$ и при этом суммарные затраты на перевозку были бы минимальны. Обозначая через $x_{ij}$ количество продукта, перевозимое из $i$-го пункта производства в $j$-го пункт потребления, приходим к следующей математической формулировке задачи:
	
	\textbf{Найти минимум целевой функции}:
	
	\[
		\sum\limits_{i=1}^n \sum\limits_{j=1}^m c_{ij}x_{ij} \longrightarrow \min \eqno(2)
	\]

	\newpage
	Так же для корректности задачи необходимо соблюдать три условия:
	
	\[
		\textbf{1.} \sum\limits_{i=1}^n x_{ij} = b_j; \ \ (j = 1, 2, ..., m) 
	\]
	
	\[
		\textbf{2.} \sum\limits_{j=1}^m x_{ij} = a_i; \ \ (i = 1, 2, ..., n)
	\]
	
	\[
		\textbf{3.} x_{ij} \geqslant 0; \ \ (i = 1, 2, ..., n; \ \ j = 1, 2, ..., m)
	\]

	Получается суммарные затраты на транспортировку в каждый пункт потребления завозится требуемое количество продукта, а так же из каждого пункта производства полностью вывозится произведенный продукт.
	
	Всякий набор величин $x_{ij} (i = 1, 2, ..., n; j = 1, 2, ..., m)$, удовлетворяющих условиям $(1-3)$, мы будем называть допустимым планом перевозок. План, для которого суммарные затраты $(2)$ достигают минимума, называется оптимальным.
	
	\begin{equation}
		X = 
		\begin{pmatrix}
			x_{11} & x_{12} & \cdots & x_{1n} \\
			x_{21} & x_{22} & \cdots & x_{2n} \\
			\vdots  & \vdots  & \ddots & \vdots  \\
			x_{m1} & x_{m2} & \cdots & x_{mn} 
		\end{pmatrix}
	\end{equation}

	\begin{flushright}
		\textit{Таблица 2.1}
	\end{flushright}
	\begin{center}
		\begin{tabular}{|c|c|c|c|c|c|c|c|}
			\hline
			\multirow{2}{*}{Поставщики} & \multicolumn{6}{c|}{Потребители} & \multirow{2}{*}{\begin{tabular}[c]{@{}c@{}}Запасы \\ поставщика\end{tabular}} \\ \cline{2-7}
			& 1 & 2 & $\cdots$ & $j$ & $\cdots$ & $m$ &  \\ \hline
			1 & $\begin{matrix} c_{11} & \\ & x_{11} \end{matrix}$ & $\begin{matrix} c_{12} & \\ & x_{12} \end{matrix}$ & $\cdots$  & $\begin{matrix} c_{1j} & \\ & x_{1j} \end{matrix}$  & $\cdots$ & $\begin{matrix} c_{1m} & \\ & x_{1m} \end{matrix}$  & $a_1$  \\ \hline
			2 & $\begin{matrix} c_{21} & \\ & x_{21} \end{matrix}$ & $\begin{matrix} c_{22} & \\ & x_{22} \end{matrix}$  & $\cdots$ & $\begin{matrix} c_{2j} & \\ & x_{2j} \end{matrix}$  & $\cdots$ & $\begin{matrix} c_{2m} & \\ & x_{2m} \end{matrix}$  & $a_2$  \\ \hline
			$\vdots$ & $\vdots$ & $\vdots$ & $\cdots$ & $\vdots$ & $\cdots$ & $\vdots$ & $\vdots$ \\ \hline
			$i$ & $\begin{matrix} c_{i1} & \\ & x_{i1} \end{matrix}$ & $\begin{matrix} c_{i2} & \\ & x_{i2} \end{matrix}$ & $\cdots$  & $\begin{matrix} c_{ij} & \\ & x_{ij} \end{matrix}$ & $\cdots$ & $\begin{matrix} c_{im} & \\ & x_{im} \end{matrix}$ & $a_i$ \\ \hline
			$\vdots$ & $\vdots$ & $\vdots$ & $\cdots$ & $\vdots$ & $\cdots$ & $\vdots$ & $\vdots$ \\ \hline
			$n$ & $\begin{matrix} c_{n1} & \\ & x_{n1} \end{matrix}$ & $\begin{matrix} c_{n2} & \\ & x_{n2} \end{matrix}$ & $\cdots$  & $\begin{matrix} c_{nj} & \\ & x_{nj} \end{matrix}$ & $\cdots$ & $\begin{matrix} c_{nm} & \\ & x_{nm} \end{matrix}$ & $a_n$  \\ \hline
			\begin{tabular}[c]{@{}c@{}}Спрос \\ потребителя\end{tabular} & $b_1$ & $b_2$ & $\cdots$ & $b_j$ & $\cdots$ & $b_m$ & $\begin{matrix} & \sum\limits_{i=1}^n a_i \\ \sum\limits_{j=1}^m b_j & \end{matrix}$ \\ \hline
		\end{tabular}
	\end{center}

	Рассмотрим теорему о разрешимости транспортной задачи:
	
	\begin{theorem}
		Транспортная задача имеет решение, если суммарный запас груза в пунктах отправления равен суммарному спросу в пунктах назначения, т.е. если выполняется равенство (1).
	\end{theorem}
	
	\begin{proof}
		В случае превышения запаса над потребностью
		$
			\sum\limits_{i=1}^n a_i > \sum\limits_{j=1}^m b_j
		$
		как уже было обговорено выше, вводится фиктивный $(m + 1)$-ый пункт назначения с потребностью
		$
			b_{m + 1} = \sum\limits_{i = 1}^n a_i - \sum\limits_{j = 1}^m b_j
		$
		
		Соответствующие тарифы считаются равными нулю: 
		
		\[
			c_{i m+1} = 0 \ \ (i=1, \cdots ,m)
		\]
		
		После этих преобразований получим закрытую модель транспортной задачи.
		
		Аналогично, при
		$
			\sum\limits_{i=1}^n a_i < \sum\limits_{j=1}^m b_j
		$
		вводится фиктивный $(n+1)$ пункт отправления с грузом,
		$
			a_{n + 1} = \sum\limits_{j = 1}^m b_j - \sum\limits_{i = 1}^n a_i
		$
		а тарифы полагаются равными нулю:
		
		\[ 
			c_{n+1 j} = 0 \ \ (j=1, \cdots ,n)
		\]
		 
		После этих преобразований получим закрытую модель транспортной задачи.
		
		Теперь если в любом случаи мы можем свести задачу к виду 
		$
			\sum\limits_{i=1}^n a_i = \sum\limits_{j=1}^m b_j = A
		$ можем получить такие величины
		
		\[
			x_{ij} = {{a_i b_j}\over{A}}; \ \ i = 1, 2, ..., n; \ \ j = 1,2, ..., m
		\]
		
		Исходя из выше полученного имеем решение:
		
		\[
			x_{ij} \geqslant 0
		\]
		
		Так как 
		$
		\sum\limits_{i=1}^n a_i = \sum\limits_{j=1}^m b_j = A
		$ получаем: 
		
		\[
			\sum\limits_{j=1}^m x_{ij} = \sum\limits_{j=1}^m {{a_i b_j}\over{A}} = 
			{{a_i \sum\limits_{j=1}^m b_j}\over{A}} = a_i
		\]
		
		\[
			\sum\limits_{i=1}^n x_{ij} = \sum\limits_{i=1}^n {{a_i b_j}\over{A}} = 
			{{b_j \sum\limits_{i=1}^n a_i}\over{A}} = b_j
		\]
		
		Следовательно, система величин $x_{ij}$, удовлетворяя всем условиям транспортной задачи, является ее решением.
	\end{proof}

	Условия транспортной задачи удобно представить в виде матрицы, которая имеет название \textbf{матрица перевозок}. В первой строке указаны величины потребностей, в первом столбце - значения запасов. В клетках внутренней матрицы ($m \times n$ штук) записывают стоимости перевозок и сами перевозки. Нумеровать будем только строки и столбцы внутренней матрицы.
	
	\textbf{Пример:}
	
	Составить математическую модель транспортной задачи перевоза груза из 3 складов в 5 магазинов. Матрица перевозок будет выглядеть так:
	\begin{flushright}
		\textit{Таблица 2.2}
	\end{flushright}
	\begin{center}
		\begin{tabular}{|c|c|c|c|c|c|c|c|c|c|c|c}
			\hline
			Пункты & \multicolumn{2}{c|}{B1} & \multicolumn{2}{c|}{B2} & \multicolumn{2}{c|}{B3} & \multicolumn{2}{c|}{B4} & \multicolumn{2}{c|}{B5} & \multicolumn{1}{c|}{Запасы} \\ \hline
			\multirow{2}{*}{A1} & \multicolumn{2}{c|}{\textbf{}} & \multicolumn{2}{c|}{} & \multicolumn{2}{c|}{} & \multicolumn{2}{c|}{} & \multicolumn{2}{c|}{} & \multicolumn{1}{c|}{\multirow{2}{*}{30}} \\ \cline{2-11}
			&  & 2 &  & 7 &  & 3 &  & 6 &  & 2 & \multicolumn{1}{c|}{} \\ \hline
			\multirow{2}{*}{A2} & \multicolumn{2}{c|}{\textbf{}} & \multicolumn{2}{c|}{\textbf{}} & \multicolumn{2}{c|}{} & \multicolumn{2}{c|}{} & \multicolumn{2}{c|}{} & \multicolumn{1}{c|}{\multirow{2}{*}{70}} \\ \cline{2-11}
			&  & 9 &  & 4 &  & 5 &  & 7 &  & 3 & \multicolumn{1}{c|}{} \\ \hline
			\multirow{2}{*}{A3} & \multicolumn{2}{c|}{} & \multicolumn{2}{c|}{\textbf{}} & \multicolumn{2}{c|}{\textbf{}} & \multicolumn{2}{c|}{} & \multicolumn{2}{c|}{} & \multicolumn{1}{c|}{\multirow{2}{*}{50}} \\ \cline{2-11}
			&  & 5 &  & 7 &  & 6 &  & 2 &  & 4 & \multicolumn{1}{c|}{} \\ \hline
			Потребность & \multicolumn{2}{c|}{10} & \multicolumn{2}{c|}{40} & \multicolumn{2}{c|}{20} & \multicolumn{2}{c|}{60} & \multicolumn{2}{c|}{20} &  \\ \cline{1-11}
		\end{tabular}
	\end{center}

	\[
		 \sum\limits_{i=1}^3 a_{i} = 30 + 70 + 50 = 150 \ \ \ \
		 \sum\limits_{j=1}^5 b_{j} = 10 + 40 + 20 + 60 + 20 = 150 
	\]
	
	В качестве примера открытой модели давайте заменим потребность $B4$, которая равняется 60 на 40. В таком случаи нужно было бы ввести еще одного потребителя с потребностью $B6 = 20$ и с нулевыми стоимостями $c_{16} = c_{26} = c_{36} = 0$. Матрица перевозок тогда станет следующей:
	
	\begin{flushright}
		\textit{Таблица 2.3}
	\end{flushright}
	\begin{center}
		\begin{tabular}{|c|c|c|c|c|c|c|c|c|c|c|c|l|c}
			\hline
			Пункты & \multicolumn{2}{c|}{B1} & \multicolumn{2}{c|}{B2} & \multicolumn{2}{c|}{B3} & \multicolumn{2}{c|}{B4} & \multicolumn{2}{c|}{B5} & \multicolumn{2}{c|}{B6} & \multicolumn{1}{c|}{Запасы} \\ \hline
			\multirow{2}{*}{A1} & \multicolumn{2}{c|}{\textbf{}} & \multicolumn{2}{c|}{} & \multicolumn{2}{c|}{} & \multicolumn{2}{c|}{} & \multicolumn{2}{c|}{} & \multicolumn{2}{c|}{} & \multicolumn{1}{c|}{\multirow{2}{*}{30}} \\ \cline{2-13}
			&  & 2 &  & 7 &  & 3 &  & 6 &  & 2 &  & 0 & \multicolumn{1}{c|}{} \\ \hline
			\multirow{2}{*}{A2} & \multicolumn{2}{c|}{\textbf{}} & \multicolumn{2}{c|}{\textbf{}} & \multicolumn{2}{c|}{} & \multicolumn{2}{c|}{} & \multicolumn{2}{c|}{} & \multicolumn{2}{c|}{} & \multicolumn{1}{c|}{\multirow{2}{*}{70}} \\ \cline{2-13}
			&  & 9 &  & 4 &  & 5 &  & 7 &  & 3 &  & 0 & \multicolumn{1}{c|}{} \\ \hline
			\multirow{2}{*}{A3} & \multicolumn{2}{c|}{} & \multicolumn{2}{c|}{\textbf{}} & \multicolumn{2}{c|}{\textbf{}} & \multicolumn{2}{c|}{} & \multicolumn{2}{c|}{} & \multicolumn{2}{c|}{} & \multicolumn{1}{c|}{\multirow{2}{*}{50}} \\ \cline{2-13}
			&  & 5 &  & 7 &  & 6 &  & 2 &  & 4 &  & 0 & \multicolumn{1}{c|}{} \\ \hline
			Потребность & \multicolumn{2}{c|}{10} & \multicolumn{2}{c|}{40} & \multicolumn{2}{c|}{20} & \multicolumn{2}{c|}{40} & \multicolumn{2}{c|}{20} & \multicolumn{2}{c|}{20} &  \\ \cline{1-13}
		\end{tabular}
	\end{center}

	\clearpage

\section{Методы построения опорного плана}
	\subsection{Метод северо-западного угла}
	\begin{theorem}
		Существует план, содержащее не более чем $(m + n - 1)$ положительных перевозок $x_{ij}$. При этом система векторов соответствующая таким перевозкам $x_{ij}$, линейно независима. 
	\end{theorem}

	\begin{proof}
		Конструктивным доказательством теоремы может послужить процесс получения первого опорного плана, предложенный Данцигом и названный Чарнесом и Купером \textbf{«правилом северо-западного угла»}. Применим это правило к следующей таблице:
		
		\begin{center}
			\begin{tabular}{cccc|c}
				$x_{11}$ & $x_{12}$ & $x_{13}$ & $x_{14}$ & $a_1$ \\
				$x_{21}$ & $x_{22}$ & $x_{23}$ & $x_{24}$ & $a_2$ \\
				$x_{31}$ & $x_{32}$ & $x_{33}$ & $x_{34}$ & $a_3$ \\ \hline
				$b_1$ & $b_2$ & $b_3$ & $b_4$ & 
			\end{tabular}
		\end{center}
	
		Определим сначала значение переменной $x_{11}$, стоящей в верхнем левом углу. Пусть $x_{11} = \min(a_1,b_1)$; если $a_1 \leqslant b_1$, то $x_{11} = a_1$ и все $x_{1j} = 0$ для $j = 2,\ 3,\ 4$. Если $a_1 \geqslant b_1$, то $x_{11} = b_1$, и все $x_{i1} = 0$ для $i = 2,\ 3,\ 4$. Для определенности допустим, что справедливо первое предположение; тогда таблица преобразуется, как показано ниже в шаге 1. Здесь общее количество продукта, вывозимого из первого пункта отправления, уменьшается до нуля, а общее количество, которое необходимо подвезти к первому пункту назначения, равно $b_1 - a_1$.
		
		\textit{Шаг 1:} Пример $b_1 > a_1$
		
		\begin{center}
			\begin{tabular}{cccc|c}
				$x_{11} = a_1$ & $0$ & $0$ & $0$ & $0$ \\
				$x_{21}$ & $x_{22}$ & $x_{23}$ & $x_{24}$ & $a_2$ \\
				$x_{31}$ & $x_{32}$ & $x_{33}$ & $x_{34}$ & $a_3$ \\ \hline
				$b_1 - a_1$ & $b_2$ & $b_3$ & $b_4$ & 
			\end{tabular}
		\end{center}
		
		После этого определяем значение первой переменной во второй строке.  Пусть $x_{21} = \min(a_2, b_1 - a_1)$. Если допустить, что $a_2 > b_1 - a_1$, $x_{21} = b_1, - а_1$, и $х_{31} = 0$. Это показано в шаге 2. Количество продукта, которое осталось перевезти из пункта отправления 2, теперь равно $a_2 - b_1 + a_1$. В свою очередь потребность первого пункта назначения полностью удовлетворена.
		
		\textit{Шаг 2:} Допустим, что $a_2 > b_1 - a_1$.
		\begin{center}
			\begin{tabular}{cccc|c}
				$x_{11} = a_1$ & $0$ & $0$ & $0$ & $0$ \\
				$x_{21} = b_1 - a_1$ & $x_{22}$ & $x_{23}$ & $x_{24}$ & $a_2 - b_1 + a_1$ \\
				$0$ & $x_{32}$ & $x_{33}$ & $x_{34}$ & $a_3$ \\ \hline
				$0$ & $b_2$ & $b_3$ & $b_4$ & 
			\end{tabular}
		\end{center}
		
		Подобным же образом в зависимости от допущений, указанных далее, получаем следующие шаги. В каждом из них определяется значение переменной $x_i$, и сводится к нулю либо запас $і$-го пункта отправления, либо потребность $j$-го пункта назначения, или и то и другое вместе.
		
		\textit{Шаг 3:} Положим $a_2 - b_1 + a_1 > b_2$.
		\begin{center}
			\begin{tabular}{cccc|c}
				$x_{11} = a_1$ & $0$ & $0$ & $0$ & $0$ \\
				$x_{21} = b_1 - a_1$ & $x_{22} = b_2$ & $x_{23}$ & $x_{24}$ & $a_2 - b_1 + a_1 - b_2$ \\
				$0$ & $0$ & $x_{33}$ & $x_{34}$ & $a_3$ \\ \hline
				$0$ & $0$ & $b_3$ & $b_4$ & 
			\end{tabular}
		\end{center}
	
		\textit{Шаг 4:} Пусть $a_2 - b_1 + a_1 - b_2 < b_3$.
		\begin{center}
			\begin{tabular}{cccc|c}
				$x_{11} = a_1$ & $0$ & $0$ & $0$ & $0$ \\
				$x_{21} = b_1 - a_1$ & $x_{22} = b_2$ & $x_{23} = a_2 - b_1 + a_1 - b_2$ & $0$ & $0$ \\
				$0$ & $0$ & $x_{33}$ & $x_{34}$ & $a_3$ \\ \hline
				$0$ & $0$ & $b_3 - a_2 + b_1 - a_1 + b_2$ & $b_4$ & 
			\end{tabular}
		\end{center}
		
		Из шага 4 видно, что $x_{33} = b_3 - a_2 + b_1 - a_2 + b_2$ и $x_{34} = b_4$. Следует отметить, что каждая из перевозок $x_i$ была получена прибавлением и вычитанием различных комбинаций $a_i$ и $b_j$. Поэтому если  $a_i$ и $b_j$ были первоначально неотрицательными целыми числами, то и решение, получаемое в результате описанного выше процесса, будет состоять из неотрицательных целых чисел. Нетрудно видеть, что этот план может содержать самое большее $n + m - 1$ положительных перевозок. При наших предположениях относительно величин  $a_i$ и $b_j$, и допущениях, сделанных при построении плана в рассмотренном примере с тремя пунктами отправления и четырьмя пунктами назначения, положительными перевозками являются:
		
		\begin{center}
			$x_{11} = a_1; \ \ x_{21} = b_1 - a_1;$ \\
			$x_{22} = b_2; \ \ x_{23} = a_2 - b_1 + a_1 - b_2;$ \\
			$x_{33} = b_3 - a_2 + b_1 - a_1 + b_2; \ \ x_{34} = b_4;$ \\
		\end{center}
		
		Используя приведенный алгоритм построения решения, можно доказать линейную независимость системы векторов, отвечающих выписанным положительным перевозкам. Тем самым будет установлено, что построенный план является опорным решением.
	\end{proof}
	
	\textbf{Пример:}
	
	Имеется транспортная таблица с исходными данными.
	\begin{center}
		\begin{tabular}{cccccccc}
			\hline
			\multicolumn{1}{|c|}{Пункты} & \multicolumn{2}{c|}{B1} & \multicolumn{2}{c|}{B2} & \multicolumn{2}{c|}{B3} & \multicolumn{1}{c|}{Запасы} \\ \hline
			\multicolumn{1}{|c|}{\multirow{2}{*}{A1}} & \multicolumn{2}{c|}{} & \multicolumn{2}{c|}{} & \multicolumn{2}{c|}{} & \multicolumn{1}{c|}{\multirow{2}{*}{10}} \\ \cline{2-7}
			\multicolumn{1}{|c|}{} & \multicolumn{1}{c|}{} & \multicolumn{1}{c|}{5} & \multicolumn{1}{c|}{} & \multicolumn{1}{c|}{3} & \multicolumn{1}{c|}{} & \multicolumn{1}{c|}{1} & \multicolumn{1}{c|}{} \\ \hline
			\multicolumn{1}{|c|}{\multirow{2}{*}{A2}} & \multicolumn{2}{c|}{} & \multicolumn{2}{c|}{} & \multicolumn{2}{c|}{} & \multicolumn{1}{c|}{\multirow{2}{*}{20}} \\ \cline{2-7}
			\multicolumn{1}{|c|}{} & \multicolumn{1}{c|}{} & \multicolumn{1}{c|}{3} & \multicolumn{1}{c|}{} & \multicolumn{1}{c|}{2} & \multicolumn{1}{c|}{} & \multicolumn{1}{c|}{4} & \multicolumn{1}{c|}{} \\ \hline
			\multicolumn{1}{|c|}{\multirow{2}{*}{A3}} & \multicolumn{2}{c|}{} & \multicolumn{2}{c|}{} & \multicolumn{2}{c|}{} & \multicolumn{1}{c|}{\multirow{2}{*}{30}} \\ \cline{2-7}
			\multicolumn{1}{|c|}{} & \multicolumn{1}{c|}{} & \multicolumn{1}{c|}{4} & \multicolumn{1}{c|}{} & \multicolumn{1}{c|}{1} & \multicolumn{1}{c|}{} & \multicolumn{1}{c|}{2} & \multicolumn{1}{c|}{} \\ \hline
			\multicolumn{1}{|c|}{Потребность} & \multicolumn{2}{c|}{15} & \multicolumn{2}{c|}{20} & \multicolumn{2}{c|}{25} &  \\ \cline{1-7}
			\multicolumn{1}{l}{} & \multicolumn{1}{l}{} & \multicolumn{1}{l}{} & \multicolumn{1}{l}{} & \multicolumn{1}{l}{} & \multicolumn{1}{l}{} & \multicolumn{1}{l}{} & \multicolumn{1}{l}{}
		\end{tabular}
	\end{center}
	
	Внесем в верхнюю левую клетку максимально возможного объема перевозки.
	
	\begin{center}
		\begin{tabular}{cccccccc}
			\hline
			\multicolumn{1}{|c|}{Пункты} & \multicolumn{2}{c|}{B1} & \multicolumn{2}{c|}{B2} & \multicolumn{2}{c|}{B3} & \multicolumn{1}{c|}{Запасы} \\ \hline
			\multicolumn{1}{|c|}{\multirow{2}{*}{A1}} & \multicolumn{2}{c|}{\textbf{10}} & \multicolumn{2}{c|}{0} & \multicolumn{2}{c|}{0} & \multicolumn{1}{c|}{\multirow{2}{*}{0}} \\ \cline{2-7}
			\multicolumn{1}{|c|}{} & \multicolumn{1}{c|}{} & \multicolumn{1}{c|}{5} & \multicolumn{1}{c|}{} & \multicolumn{1}{c|}{3} & \multicolumn{1}{c|}{} & \multicolumn{1}{c|}{1} & \multicolumn{1}{c|}{} \\ \hline
			\multicolumn{1}{|c|}{\multirow{2}{*}{A2}} & \multicolumn{2}{c|}{} & \multicolumn{2}{c|}{} & \multicolumn{2}{c|}{} & \multicolumn{1}{c|}{\multirow{2}{*}{20}} \\ \cline{2-7}
			\multicolumn{1}{|c|}{} & \multicolumn{1}{c|}{} & \multicolumn{1}{c|}{3} & \multicolumn{1}{c|}{} & \multicolumn{1}{c|}{2} & \multicolumn{1}{c|}{} & \multicolumn{1}{c|}{4} & \multicolumn{1}{c|}{} \\ \hline
			\multicolumn{1}{|c|}{\multirow{2}{*}{A3}} & \multicolumn{2}{c|}{} & \multicolumn{2}{c|}{} & \multicolumn{2}{c|}{} & \multicolumn{1}{c|}{\multirow{2}{*}{30}} \\ \cline{2-7}
			\multicolumn{1}{|c|}{} & \multicolumn{1}{c|}{} & \multicolumn{1}{c|}{4} & \multicolumn{1}{c|}{} & \multicolumn{1}{c|}{1} & \multicolumn{1}{c|}{} & \multicolumn{1}{c|}{2} & \multicolumn{1}{c|}{} \\ \hline
			\multicolumn{1}{|c|}{Потребность} & \multicolumn{2}{c|}{5} & \multicolumn{2}{c|}{20} & \multicolumn{2}{c|}{25} &  \\ \cline{1-7}
			\multicolumn{1}{l}{} & \multicolumn{1}{l}{} & \multicolumn{1}{l}{} & \multicolumn{1}{l}{} & \multicolumn{1}{l}{} & \multicolumn{1}{l}{} & \multicolumn{1}{l}{} & \multicolumn{1}{l}{}
		\end{tabular}
	\end{center}

	Запасы на складе $A1$ закончились, поэтому в оставшиеся ячейки данной строки ставим прочерки. Затем переходим к следующей строке и заполняем ее ячейки слева направо.
	
	\begin{center}
		\begin{tabular}{|c|c|c|c|c|c|c|c}
			\hline
			Пункты & \multicolumn{2}{c|}{B1} & \multicolumn{2}{c|}{B2} & \multicolumn{2}{c|}{B3} & \multicolumn{1}{c|}{Запасы} \\ \hline
			\multirow{2}{*}{A1} & \multicolumn{2}{c|}{\textbf{10}} & \multicolumn{2}{c|}{0} & \multicolumn{2}{c|}{0} & \multicolumn{1}{c|}{\multirow{2}{*}{0}} \\ \cline{2-7}
			&  & 5 &  & 3 &  & 1 & \multicolumn{1}{c|}{} \\ \hline
			\multirow{2}{*}{A2} & \multicolumn{2}{c|}{\textbf{5}} & \multicolumn{2}{c|}{} & \multicolumn{2}{c|}{} & \multicolumn{1}{c|}{\multirow{2}{*}{15}} \\ \cline{2-7}
			&  & 3 &  & 2 &  & 4 & \multicolumn{1}{c|}{} \\ \hline
			\multirow{2}{*}{A3} & \multicolumn{2}{c|}{0} & \multicolumn{2}{c|}{} & \multicolumn{2}{c|}{} & \multicolumn{1}{c|}{\multirow{2}{*}{30}} \\ \cline{2-7}
			&  & 4 &  & 1 &  & 2 & \multicolumn{1}{c|}{} \\ \hline
			Потребность & \multicolumn{2}{c|}{0} & \multicolumn{2}{c|}{20} & \multicolumn{2}{c|}{25} &  \\ \cline{1-7}
		\end{tabular}
	\end{center}
	
	\begin{center}
		\begin{tabular}{|c|c|c|c|c|c|c|c}
			\hline
			Пункты & \multicolumn{2}{c|}{B1} & \multicolumn{2}{c|}{B2} & \multicolumn{2}{c|}{B3} & \multicolumn{1}{c|}{Запасы} \\ \hline
			\multirow{2}{*}{A1} & \multicolumn{2}{c|}{\textbf{10}} & \multicolumn{2}{c|}{0} & \multicolumn{2}{c|}{0} & \multicolumn{1}{c|}{\multirow{2}{*}{0}} \\ \cline{2-7}
			&  & 5 &  & 3 &  & 1 & \multicolumn{1}{c|}{} \\ \hline
			\multirow{2}{*}{A2} & \multicolumn{2}{c|}{\textbf{5}} & \multicolumn{2}{c|}{\textbf{15}} & \multicolumn{2}{c|}{0} & \multicolumn{1}{c|}{\multirow{2}{*}{0}} \\ \cline{2-7}
			&  & 3 &  & 2 &  & 4 & \multicolumn{1}{c|}{} \\ \hline
			\multirow{2}{*}{A3} & \multicolumn{2}{c|}{0} & \multicolumn{2}{c|}{} & \multicolumn{2}{c|}{} & \multicolumn{1}{c|}{\multirow{2}{*}{30}} \\ \cline{2-7}
			&  & 4 &  & 1 &  & 2 & \multicolumn{1}{c|}{} \\ \hline
			Потребность & \multicolumn{2}{c|}{0} & \multicolumn{2}{c|}{5} & \multicolumn{2}{c|}{25} &  \\ \cline{1-7}
		\end{tabular}
	\end{center}
	
	Переходим к третьей строке и тоже заполняем ее слева направо.
	
	\begin{center}
		\begin{tabular}{|c|c|c|c|c|c|c|c}
			\hline
			Пункты & \multicolumn{2}{c|}{B1} & \multicolumn{2}{c|}{B2} & \multicolumn{2}{c|}{B3} & \multicolumn{1}{c|}{Запасы} \\ \hline
			\multirow{2}{*}{A1} & \multicolumn{2}{c|}{\textbf{10}} & \multicolumn{2}{c|}{0} & \multicolumn{2}{c|}{0} & \multicolumn{1}{c|}{\multirow{2}{*}{0}} \\ \cline{2-7}
			&  & 5 &  & 3 &  & 1 & \multicolumn{1}{c|}{} \\ \hline
			\multirow{2}{*}{A2} & \multicolumn{2}{c|}{\textbf{5}} & \multicolumn{2}{c|}{\textbf{15}} & \multicolumn{2}{c|}{0} & \multicolumn{1}{c|}{\multirow{2}{*}{0}} \\ \cline{2-7}
			&  & 3 &  & 2 &  & 4 & \multicolumn{1}{c|}{} \\ \hline
			\multirow{2}{*}{A3} & \multicolumn{2}{c|}{0} & \multicolumn{2}{c|}{\textbf{5}} & \multicolumn{2}{c|}{} & \multicolumn{1}{c|}{\multirow{2}{*}{25}} \\ \cline{2-7}
			&  & 4 &  & 1 &  & 2 & \multicolumn{1}{c|}{} \\ \hline
			Потребность & \multicolumn{2}{c|}{0} & \multicolumn{2}{c|}{0} & \multicolumn{2}{c|}{25} &  \\ \cline{1-7}
		\end{tabular}
	\end{center}

	\begin{center}
		\begin{tabular}{|c|c|c|c|c|c|c|c}
			\hline
			Пункты & \multicolumn{2}{c|}{B1} & \multicolumn{2}{c|}{B2} & \multicolumn{2}{c|}{B3} & \multicolumn{1}{c|}{Запасы} \\ \hline
			\multirow{2}{*}{A1} & \multicolumn{2}{c|}{\textbf{10}} & \multicolumn{2}{c|}{0} & \multicolumn{2}{c|}{0} & \multicolumn{1}{c|}{\multirow{2}{*}{0}} \\ \cline{2-7}
			&  & 5 &  & 3 &  & 1 & \multicolumn{1}{c|}{} \\ \hline
			\multirow{2}{*}{A2} & \multicolumn{2}{c|}{\textbf{5}} & \multicolumn{2}{c|}{\textbf{15}} & \multicolumn{2}{c|}{0} & \multicolumn{1}{c|}{\multirow{2}{*}{0}} \\ \cline{2-7}
			&  & 3 &  & 2 &  & 4 & \multicolumn{1}{c|}{} \\ \hline
			\multirow{2}{*}{A3} & \multicolumn{2}{c|}{0} & \multicolumn{2}{c|}{\textbf{5}} & \multicolumn{2}{c|}{\textbf{25}} & \multicolumn{1}{c|}{\multirow{2}{*}{0}} \\ \cline{2-7}
			&  & 4 &  & 1 &  & 2 & \multicolumn{1}{c|}{} \\ \hline
			Потребность & \multicolumn{2}{c|}{0} & \multicolumn{2}{c|}{0} & \multicolumn{2}{c|}{0} &  \\ \cline{1-7}
		\end{tabular}
	\end{center}

	Получено опорное решение:
	\begin{equation}
		X = 
		\begin{pmatrix}
			10 & 0 & 0 \\
			5 & 15 & 0 \\
			0 & 5 & 25 \\ 
		\end{pmatrix}
	\end{equation}
	\clearpage
	
	\subsection{Метод наименьшего элемента}
	Ключевая идея заключается в следующем: определяем ячейку транспортной таблицы с наименьшим значением тарифа на перевозку груза (если окажется, что есть несколько ячеек с одинаковыми и минимальными тарифами — выбираем любую из них).
	
	В эту ячейку выписываем максимально возможный объем груза, который можно доставить с соответствующего этой ячейке склада на соответствующий завод.
	
	\begin{center}
		\begin{tabular}{|c|c|c|c|c|c|c|c}
			\hline
			Пункты & \multicolumn{2}{c|}{B1} & \multicolumn{2}{c|}{B2} & \multicolumn{2}{c|}{B3} & \multicolumn{1}{c|}{Запасы} \\ \hline
			\multirow{2}{*}{A1} & \multicolumn{2}{c|}{\textbf{}} & \multicolumn{2}{c|}{} & \multicolumn{2}{c|}{} & \multicolumn{1}{c|}{\multirow{2}{*}{10}} \\ \cline{2-7}
			&  & 5 &  & 3 &  & 1 & \multicolumn{1}{c|}{} \\ \hline
			\multirow{2}{*}{A2} & \multicolumn{2}{c|}{\textbf{}} & \multicolumn{2}{c|}{\textbf{}} & \multicolumn{2}{c|}{} & \multicolumn{1}{c|}{\multirow{2}{*}{20}} \\ \cline{2-7}
			&  & 3 &  & 2 &  & 4 & \multicolumn{1}{c|}{} \\ \hline
			\multirow{2}{*}{A3} & \multicolumn{2}{c|}{} & \multicolumn{2}{c|}{\textbf{}} & \multicolumn{2}{c|}{\textbf{}} & \multicolumn{1}{c|}{\multirow{2}{*}{30}} \\ \cline{2-7}
			&  & 4 &  & 1 &  & 2 & \multicolumn{1}{c|}{} \\ \hline
			Потребность & \multicolumn{2}{c|}{15} & \multicolumn{2}{c|}{20} & \multicolumn{2}{c|}{25} &  \\ \cline{1-7}
		\end{tabular}
	\end{center}

	Объемы запасов и потребностей уменьшаются на величину груза. Если запасы склада исчерпаны, то полностью вычеркиваем эту строку таблицы. Если потребности завода полностью удовлетворены — полностью вычеркиваем этот столбец таблицы.
	
	\begin{center}
		\begin{tabular}{|c|c|c|c|c|c|c|c}
			\hline
			Пункты & \multicolumn{2}{c|}{B1} & \multicolumn{2}{c|}{B2} & \multicolumn{2}{c|}{B3} & \multicolumn{1}{c|}{Запасы} \\ \hline
			\multirow{2}{*}{A1} & \multicolumn{2}{c|}{0} & \multicolumn{2}{c|}{0} & \multicolumn{2}{c|}{10} & \multicolumn{1}{c|}{\multirow{2}{*}{0}} \\ \cline{2-7}
			&  & 5 &  & 3 &  & \textbf{1} & \multicolumn{1}{c|}{} \\ \hline
			\multirow{2}{*}{A2} & \multicolumn{2}{c|}{\textbf{}} & \multicolumn{2}{c|}{\textbf{}} & \multicolumn{2}{c|}{} & \multicolumn{1}{c|}{\multirow{2}{*}{20}} \\ \cline{2-7}
			&  & 3 &  & 2 &  & 4 & \multicolumn{1}{c|}{} \\ \hline
			\multirow{2}{*}{A3} & \multicolumn{2}{c|}{} & \multicolumn{2}{c|}{} & \multicolumn{2}{c|}{\textbf{}} & \multicolumn{1}{c|}{\multirow{2}{*}{30}} \\ \cline{2-7}
			&  & 4 &  & 1 &  & 2 & \multicolumn{1}{c|}{} \\ \hline
			Потребность & \multicolumn{2}{c|}{15} & \multicolumn{2}{c|}{20} & \multicolumn{2}{c|}{25} &  \\ \cline{1-7}
		\end{tabular}
	\end{center}

	Продолжаем в том же духе до тех пор, пока все запасы не будут исчерпаны, а все потребности удовлетворены.
	
	\begin{center}
		\begin{tabular}{|c|c|c|c|c|c|c|c}
			\hline
			Пункты & \multicolumn{2}{c|}{B1} & \multicolumn{2}{c|}{B2} & \multicolumn{2}{c|}{B3} & \multicolumn{1}{c|}{Запасы} \\ \hline
			\multirow{2}{*}{A1} & \multicolumn{2}{c|}{0} & \multicolumn{2}{c|}{0} & \multicolumn{2}{c|}{10} & \multicolumn{1}{c|}{\multirow{2}{*}{0}} \\ \cline{2-7}
			&  & 5 &  & 3 &  & 1 & \multicolumn{1}{c|}{} \\ \hline
			\multirow{2}{*}{A2} & \multicolumn{2}{c|}{} & \multicolumn{2}{c|}{0} & \multicolumn{2}{c|}{\textbf{}} & \multicolumn{1}{c|}{\multirow{2}{*}{20}} \\ \cline{2-7}
			&  & 3 &  & 2 &  & 4 & \multicolumn{1}{c|}{} \\ \hline
			\multirow{2}{*}{A3} & \multicolumn{2}{c|}{} & \multicolumn{2}{c|}{20} & \multicolumn{2}{c|}{\textbf{}} & \multicolumn{1}{c|}{\multirow{2}{*}{10}} \\ \cline{2-7}
			&  & 4 &  & \textbf{1} &  & 2 & \multicolumn{1}{c|}{} \\ \hline
			Потребность & \multicolumn{2}{c|}{15} & \multicolumn{2}{c|}{0} & \multicolumn{2}{c|}{25} &  \\ \cline{1-7}
		\end{tabular}
	\end{center}

	\begin{center}
		\begin{tabular}{|c|c|c|c|c|c|c|c}
			\hline
			Пункты & \multicolumn{2}{c|}{B1} & \multicolumn{2}{c|}{B2} & \multicolumn{2}{c|}{B3} & \multicolumn{1}{c|}{Запасы} \\ \hline
			\multirow{2}{*}{A1} & \multicolumn{2}{c|}{0} & \multicolumn{2}{c|}{0} & \multicolumn{2}{c|}{10} & \multicolumn{1}{c|}{\multirow{2}{*}{0}} \\ \cline{2-7}
			&  & 5 &  & 3 &  & 1 & \multicolumn{1}{c|}{} \\ \hline
			\multirow{2}{*}{A2} & \multicolumn{2}{c|}{} & \multicolumn{2}{c|}{0} & \multicolumn{2}{c|}{\textbf{}} & \multicolumn{1}{c|}{\multirow{2}{*}{20}} \\ \cline{2-7}
			&  & 3 &  & 2 &  & 4 & \multicolumn{1}{c|}{} \\ \hline
			\multirow{2}{*}{A3} & \multicolumn{2}{c|}{0} & \multicolumn{2}{c|}{20} & \multicolumn{2}{c|}{10} & \multicolumn{1}{c|}{\multirow{2}{*}{0}} \\ \cline{2-7}
			&  & 4 &  & 1 &  & \textbf{2} & \multicolumn{1}{c|}{} \\ \hline
			Потребность & \multicolumn{2}{c|}{15} & \multicolumn{2}{c|}{0} & \multicolumn{2}{c|}{5} &  \\ \cline{1-7}
		\end{tabular}
	\end{center}

	\begin{center}
		\begin{tabular}{|c|c|c|c|c|c|c|c}
			\hline
			Пункты & \multicolumn{2}{c|}{B1} & \multicolumn{2}{c|}{B2} & \multicolumn{2}{c|}{B3} & \multicolumn{1}{c|}{Запасы} \\ \hline
			\multirow{2}{*}{A1} & \multicolumn{2}{c|}{0} & \multicolumn{2}{c|}{0} & \multicolumn{2}{c|}{10} & \multicolumn{1}{c|}{\multirow{2}{*}{0}} \\ \cline{2-7}
			&  & 5 &  & 3 &  & 1 & \multicolumn{1}{c|}{} \\ \hline
			\multirow{2}{*}{A2} & \multicolumn{2}{c|}{15} & \multicolumn{2}{c|}{0} & \multicolumn{2}{c|}{\textbf{}} & \multicolumn{1}{c|}{\multirow{2}{*}{5}} \\ \cline{2-7}
			&  & \textbf{3} &  & 2 &  & 4 & \multicolumn{1}{c|}{} \\ \hline
			\multirow{2}{*}{A3} & \multicolumn{2}{c|}{0} & \multicolumn{2}{c|}{20} & \multicolumn{2}{c|}{10} & \multicolumn{1}{c|}{\multirow{2}{*}{0}} \\ \cline{2-7}
			&  & 4 &  & 1 &  & 2 & \multicolumn{1}{c|}{} \\ \hline
			Потребность & \multicolumn{2}{c|}{0} & \multicolumn{2}{c|}{0} & \multicolumn{2}{c|}{5} &  \\ \cline{1-7}
		\end{tabular}
	\end{center}

	\begin{center}
		\begin{tabular}{|c|c|c|c|c|c|c|c}
			\hline
			Пункты & \multicolumn{2}{c|}{B1} & \multicolumn{2}{c|}{B2} & \multicolumn{2}{c|}{B3} & \multicolumn{1}{c|}{Запасы} \\ \hline
			\multirow{2}{*}{A1} & \multicolumn{2}{c|}{0} & \multicolumn{2}{c|}{0} & \multicolumn{2}{c|}{10} & \multicolumn{1}{c|}{\multirow{2}{*}{0}} \\ \cline{2-7}
			&  & 5 &  & 3 &  & 1 & \multicolumn{1}{c|}{} \\ \hline
			\multirow{2}{*}{A2} & \multicolumn{2}{c|}{15} & \multicolumn{2}{c|}{0} & \multicolumn{2}{c|}{5} & \multicolumn{1}{c|}{\multirow{2}{*}{0}} \\ \cline{2-7}
			&  & 3 &  & 2 &  & \textbf{4} & \multicolumn{1}{c|}{} \\ \hline
			\multirow{2}{*}{A3} & \multicolumn{2}{c|}{0} & \multicolumn{2}{c|}{20} & \multicolumn{2}{c|}{10} & \multicolumn{1}{c|}{\multirow{2}{*}{0}} \\ \cline{2-7}
			&  & 4 &  & 1 &  & 2 & \multicolumn{1}{c|}{} \\ \hline
			Потребность & \multicolumn{2}{c|}{0} & \multicolumn{2}{c|}{0} & \multicolumn{2}{c|}{0} &  \\ \cline{1-7}
		\end{tabular}
	\end{center}

	В итоге мы получим опорный план перевозок для транспортной задачи.
	
	\begin{equation}
		X = 
		\begin{pmatrix}
			0 & 0 & 10 \\
			15 & 0 & 5 \\
			0 & 20 & 10 \\ 
		\end{pmatrix}
	\end{equation}
	\clearpage
	
\section{Проверка опорного плана на вырожденность}
	Как известно, транспортная задача бывает открытая и закрытая. Однако на практике чаще всего попадается открытый тип транспортной задачи. Способ борьбы с этим уже был рассмотрен.
	
	Так же известны и другие виды задачи - вырожденной и невырожденной.
	
	Вырожденной является такая задача, в которой количество клеток в плане меньше $m+n-1$. Способ борьбы с ним довольно много: от случайно размещаемых нулевых клеток в плане до перебора по очереди всех пустых клеток. Однако такой путь не дает гарантий в том, что вновь размещенная клетка не образует \textbf{цикл} с имеющимися клетками.
	
	\clearpage
	
\section{Методы решения}
	\subsection{Метод северо-западного угла}
	
	\clearpage
	
	\subsection{Метод наименьшего элемента}
	
	\clearpage
	
	\subsection{Метод потенциалов}
	
	Метод состоит из конечного числа итераций, с помощью которых по некоторому исходному плану задачи строится ее оптимальное решение. Каждая итерация разбивается на $2$ этапа. На $1$-ом этапе план, полученный в результате предыдущей итерации, проверяется на оптимальность. Если план оказывается решением, процесс заканчивается. Если же это не так, осуществляется переход к этапу $2$. На $2$-ом этапе строится новый план, приводящий к меньшим (по сравнению с предыдущим планом) транспортным издержкам. Перед тем, как начать детальное рассмотрение метода, введем несколько необходимых понятий.
	
	\clearpage

\section{Решение с помощью теории графов}

	\clearpage

\begin{thebibliography}{99}
	\addcontentsline{toc}{section}{Список литературы}
	\bibitem{1}
	Юдин Д.Б., Гольштейн Е.Г. Задачи и методы линейного программирования;
	Москва: Советское радио, 1969 - 736с.
	\bibitem{2}
	Канторович Л.В., Горстко А.Б. Математическое оптимальное программирование  в экономике;
	Москва: Знание, 1968 - 96с.
	\bibitem{3}
	Палий И.А., Линейное программирование. Учебное пособие;
	Москва: Эксмо, 2008 - 256с.
	\bibitem{4} 
	Костевич Л.C. Математическое программирование. Информационные технологии оптимальных решений;
	Минск: Новое знание, 2003 - 424c.
	\bibitem{5}
	Акулич И.Л. Математическое программирование в примерах и задачах; Санкт-Петербург: Лань, 2011 - 352с.
	\bibitem{6}
	Данциг Д. Линейное программирование, его обобщения и применения; Москва: Прогресс, 1966 - 602с.
	\bibitem{7}
	Гасс С. Линейное программирование; Москва: Физматгиз, 1961 - 303с.
\end{thebibliography}

\end{document} 