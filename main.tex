\documentclass[a4paper,12pt]{article}
\usepackage[T2A]{fontenc}
\usepackage[utf8]{inputenc}
\usepackage[main = russian,english]{babel}
\usepackage[14pt]{extsizes}
\usepackage{cmap}
\usepackage{indentfirst}
\usepackage{autonum}
\usepackage{amsfonts}
\usepackage{amsmath}
\usepackage{amssymb}
\usepackage{amsthm}
\usepackage{upgreek}
\usepackage{graphicx}
\usepackage{listings}
\usepackage{setspace,amsmath}
\usepackage[table,xcdraw]{xcolor}
\usepackage[left=15mm, top=20mm, right=15mm, bottom=20mm, nohead, footskip=15mm]{geometry}

\begin{document}
\setcounter{page}{0}

\begin{center}
	\small{Министерство науки и высшего образования Российской Федерации}\\
	\small{Федеральное государственное бюджетное образовательное учреждение}\\
	\small{Высшего образования}\\
	\small{\textbf{«Северо-Осетинский государственный университет\\
			имени Коста Левановича Хетагурова»}}\\
		
	\hfill \break
	\hfill \break
	\hfill \break
	\hfill \break
	\hfill \break
	\hfill \break
	\hfill \break
	\hfill \break
	\hfill \break
	
	\normalsize{Курсовая работа}\\
	\large{\textbf{«Транспортная задача. Методы решенияю.»}}\\
	
	\hfill \break
	\hfill \break
	\hfill \break
	\hfill \break
	\hfill \break
	\hfill\break
\end{center}

\begin{flushright}
	\textbf{Выполнил:}\\
	Студент 3 курса направления:\\
	«Прикладная математика и информатика»\\
	\textit{Гамосов Станислав Станиславович}\\
\end{flushright}

\hfill

\begin{flushright}
	\textbf{Научный руководитель:}\\
	кандидат физико-математических наук\\
	\textit{Тотиева Жанна Дмитриевна}\\
\end{flushright}

\hfill

\begin{flushright}
	\textbf{«Работа допустима к защите»}\\
	Заведующий кафедрой\\
	доктор физико-математических наук\\
	\textit{Кусраев. А.Г. \underline{\hspace{3cm}}}\\
\end{flushright}

\normalsize{ \hspace{28pt}} \hfill \break
\begin{center} Владикавказ 2021 \end{center}
\thispagestyle{empty}


\tableofcontents{}
\clearpage

\section{Введение}
	\textbf{Транспортная задача} – это спектр задач с единой математической моделью, классическая формулировка, которой звучит: \textit{«Задача о наиболее экономном плане перевозок однородного продукта или взаимозаменяемых продуктов из пунктов производства в пункты потребления»}. Такая форма встречается чаще всего в линейном программирование, а если точнее в его практических приложениях. 
	
	\textbf{Линейное программирование} является одним из разделов математического программирования – области математики, разрабатывающей теорию и численные методы решения многомерных экстремальных задач с ограничениями.
	
	Проблема была впервые формализована французским математиком \textit{Гаспаром Монжем} в 1781 году. Прогресс в решении проблемы был достигнут во время Великой Отечественной войны советским математиком и экономистом \textit{Леонидом Канторовичем}. Поэтому иногда эта проблема называется \textbf{транспортной задачей Монжа — Канторовича}.
	
	Если вернуться к самой задачи огромное количество возможных вариантов перевозок затрудняет получение достаточно экономного плана эмпирическим или экспертным путем. Применение математических методов и вычислительных в планировании перевозок дает большой экономический эффект. Транспортные задачи могут быть решены \textbf{симплексным методом} однако матрица системы ограничений транспортной задачи настолько своеобразна, что для ее решения разработаны специальные методы. Они, как и \textbf{симплексный метод}, позволяют найти начальное опорное решение, а затем, улучшая его получить оптимальный результат. Транспортная задача может
	также решаться с ограничениями и без ограничений.
	
	В зависимости от способа представления условий транспортной задачи она может быть представлена в \textbf{графовой} или \textbf{матричной} форме.
	
	\newpage
\section{Постановка задачи}
	
	Задача эта возникает, когда речь идет о рациональной перевозке некоторого однородного продукта от производителей к потребителям. В этом случае для каждого потребителя безразлично, откуда, из каких пунктов производства будет поступать этот продукт, лишь бы он поступал в нужном объеме. Однако от того, насколько рациональным будет прикрепление пунктов потребления к пунктам производства, существенно зависит объем транспортной работы. В связи с этим естественно возникает вопрос о наиболее эффективном прикреплении, правильном направлении перевозок груза, при котором потребности удовлетворяются, а затраты на транспортировку минимальны. Более точно задача формулируется так.
	
	Пусть имеются пункты производства $(A_1, A_2, ...,  A_n)$ с объемами производства в единицу времени, равными соответственно $(a_1, a_2, ...,  a_n)$, и пункты потребления $(B_1, B_2, ...,  B_m)$ с объемами потребления, равными $(b_1, b_2, ..., b_m)$ соответственно. Будем предполагать, что производство и потребление сбалансированы — сумма объемов производства равна сумме объемов потребления. Такой вид транспортной задачи называется \textbf{закрытым}.
	
	\[
		\sum\limits_{i=1}^n a_i = \sum\limits_{j=1}^m b_j \eqno(1)
	\]
	
	В дальнейшем будем рассматривать только такой тип задачи. Однако любую открытую транспортную задачу ($\sum\limits_{i=1}^n a_i \neq \sum\limits_{j=1}^m b_j$) легко закрыть. Нужно ввести дополнительный пункт производства (пункт потребления) с недостающим объемом производства (объемом потребления) и с нулевыми стоимостями перевозок.

	Предполагается, что известны величины $c_{ij}$ — затраты по перевозке единицы продукта из $i$-го пункта производства в $j$-й пункт потребления. Они могут быть выражены в стоимостной (денежной) форме или в натуральной (километрах). Требуется найти такой план перевозок, при котором были бы удовлетворены потребности в пунктах $(B_1, B_2, ...,  B_m)$ и при этом суммарные затраты на перевозку были бы минимальны. Обозначая через $x_{ij}$ количество продукта, перевозимое из $i$-го пункта производства в $j$-го пункт потребления, приходим к следующей математической формулировке задачи:
	
	Найти минимум
	
	\[
		\sum\limits_{i=1}^n \sum\limits_{j=1}^m c_{ij}x_{ij} \eqno(2)
	\]

	\newpage
	Так же для корректности задачи необходимо соблюдать три условия:
	
	\[
		\textbf{1.} \sum\limits_{i=1}^n x_{ij} = b_j; (j = 1, 2, ..., m) 
	\]
	
	\[
		\textbf{2.} \sum\limits_{j=1}^m x_{ij} = a_i; (i = 1, 2, ..., n)
	\]
	
	\[
		\textbf{3.} x_{ij} \geqslant 0; (i = 1, 2, ..., n; j = 1, 2, ..., m)
	\]

	Получается суммарные затраты на транспортировку в каждый пункт потребления завозится требуемое количество продукта, а так же из каждого пункта производства полностью вывозится произведенный продукт.
	
	Всякий набор величин $x_{ij} (i = 1, 2, ..., n; j = 1, 2, ..., m)$, удовлетворяющих условиям $(1-3)$, мы будем называть допустимым планом перевозок. План, для которого суммарные затраты $(2)$ достигают минимума, называется оптимальным.
	
	Полученные условия транспортную задачи удобно представить в виде матрицы, которая имеет название \textbf{матрица перевозок}. В ней $(m + 1)$ строк и $(n + 1)$ столбец. В первой строке указаны величины потребностей, в первом столбце - значения запасов. В клетках внутренней матрицы ($m \times n$ штук) записывают стоимости перевозок и сами перевозки. Нумеровать будем только строки и столбцы внутренней матрицы.
	
	\begin{flushright}
		\textit{Таблица 2.1}
	\end{flushright}
	\begin{center}
		\begin{tabular}{c|cccccc}
			\cline{2-7}
			& \multicolumn{1}{c|}{$b_j$} & \multicolumn{1}{c|}{10} & \multicolumn{1}{c|}{40} & \multicolumn{1}{c|}{20} & \multicolumn{1}{c|}{60} & \multicolumn{1}{c|}{20} \\ \hline
			\multicolumn{1}{|c|}{$a_i$} &  &  &  &  &  &  \\ \cline{1-1} \cline{3-7} 
			\multicolumn{1}{|c|}{30} & \multicolumn{1}{c|}{} & \multicolumn{1}{c|}{2} & \multicolumn{1}{c|}{7} & \multicolumn{1}{c|}{3} & \multicolumn{1}{c|}{6} & \multicolumn{1}{c|}{2} \\ \cline{1-1} \cline{3-7} 
			\multicolumn{1}{|c|}{70} & \multicolumn{1}{c|}{} & \multicolumn{1}{c|}{9} & \multicolumn{1}{c|}{4} & \multicolumn{1}{c|}{5} & \multicolumn{1}{c|}{7} & \multicolumn{1}{c|}{3} \\ \cline{1-1} \cline{3-7} 
			\multicolumn{1}{|c|}{50} & \multicolumn{1}{c|}{} & \multicolumn{1}{c|}{5} & \multicolumn{1}{c|}{7} & \multicolumn{1}{c|}{6} & \multicolumn{1}{c|}{2} & \multicolumn{1}{c|}{4} \\ \cline{1-1} \cline{3-7} 
		\end{tabular}
	\end{center}

	\[
		 \sum\limits_{i=1}^3 a_{i} = 30 + 70 + 50 = 150 \ \ \ \
		 \sum\limits_{j=1}^5 b_{j} = 10 + 40 + 20 + 60 + 20 = 150 
	\]
	
	Если бы, например, потребность $b_4$, равнялась не 60, а 40, нужно было бы ввести еще одного потребителя с потребностью $b_6 = 20$ и с нулевыми стоимостями $c_{16} = c_{26} = c_{36} = 0$. Матрица перевозок тогда станет следующей:
	
	\begin{flushright}
		\textit{Таблица 2.2}
	\end{flushright}
	\begin{center}
		\begin{tabular}{c|ccccccc}
			\cline{2-8}
			& \multicolumn{1}{c|}{$b_j$} & \multicolumn{1}{c|}{10} & \multicolumn{1}{c|}{40} & \multicolumn{1}{c|}{20} & \multicolumn{1}{c|}{40} & \multicolumn{1}{c|}{20} & \multicolumn{1}{l|}{20} \\ \hline
			\multicolumn{1}{|c|}{$a_i$} &  &  &  &  &  &  & \multicolumn{1}{l}{} \\ \cline{1-1} \cline{3-8} 
			\multicolumn{1}{|c|}{30} & \multicolumn{1}{c|}{} & \multicolumn{1}{c|}{2} & \multicolumn{1}{c|}{7} & \multicolumn{1}{c|}{3} & \multicolumn{1}{c|}{6} & \multicolumn{1}{c|}{2} & \multicolumn{1}{c|}{0} \\ \cline{1-1} \cline{3-8} 
			\multicolumn{1}{|c|}{70} & \multicolumn{1}{c|}{} & \multicolumn{1}{c|}{9} & \multicolumn{1}{c|}{4} & \multicolumn{1}{c|}{5} & \multicolumn{1}{c|}{7} & \multicolumn{1}{c|}{3} & \multicolumn{1}{c|}{0} \\ \cline{1-1} \cline{3-8} 
			\multicolumn{1}{|c|}{50} & \multicolumn{1}{c|}{} & \multicolumn{1}{c|}{5} & \multicolumn{1}{c|}{7} & \multicolumn{1}{c|}{6} & \multicolumn{1}{c|}{2} & \multicolumn{1}{c|}{4} & \multicolumn{1}{c|}{0} \\ \cline{1-1} \cline{3-8} 
		\end{tabular}
	\end{center}

	Запишем математическую модель первой из указанных задач:
	
	\begin{equation}
		\begin{cases}
			Z = \sum\limits_{i=1}^n \sum\limits_{j=1}^m c_{ij}x_{ij} \longrightarrow \min \\
			x_{11} + x_{12} + x_{13} + x_{14} + x_{15} = 30\\
			x_{21} + x_{22} + x_{23} + x_{24} + x_{25} = 70\\
			x_{31} + x_{32} + x_{33} + x_{34} + x_{35} = 50\\
			x_{11} + x_{21} + x_{31} = 10\\
			x_{12} + x_{22} + x_{32} = 40\\
			x_{13} + x_{23} + x_{33} = 20\\
			x_{14} + x_{24} + x_{34} = 60\\
			x_{15} + x_{25} + x_{35} = 20\\
			x_{11} + x_{21} + x_{31} = 10\\ 
			\forall (i,j): x_{ij} \geqslant 0
		\end{cases}
	\end{equation}
	\newpage
	
	\begin{thebibliography}{3}
		\bibitem{1}
		Юдин Д.Б., Гольштейн Е.Г. Задачи и методы линейного
		программирования. – Москва: Советское радио, 1969. – 736с.
		\bibitem{2}
		Канторович Л.В., Горстко А.Б. Математическое оптимальное программирование 
		в экономике - Москва: Знание, 1968. - 96с.
		\bibitem{3}
		Палий И.А., Линейное программирование. Учебное пособие Москва: Эксмо, 2008. - 256 с.
		\bibitem{4}
		\bibitem{5}
		\bibitem{6}
		\bibitem{7}
	\end{thebibliography}

\end{document} 